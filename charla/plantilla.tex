\documentclass[aspectratio=169,10pt,xcolor=svgnames,compress]{beamer} 
% Explicación de las opciones:
% aspectratio es para que tenga formato 16:9
% 10pt es el tamaño de la fuente por defecto
% xcolor=svgnames selecciona el esquema de colores a utilizar
% compress trata de achicar todas las barras de navegación
% Agregar handout a las opciones, para tener una versión imprimible.

\input{tex/diapo_encabezado.tex}
\input{tex/tikzlibrarybayesnet.code.tex}
\newif\ifen
\newif\ifes
\newcommand{\en}[1]{\ifen#1\fi}
\newcommand{\es}[1]{\ifes#1\fi}
\entrue
\newcommand{\E}{\en{S}\es{E}}
\newcommand{\A}{\en{E}\es{A}}
\newcommand{\Ee}{\en{s}\es{e}}
\newcommand{\Aa}{\en{e}\es{a}}
% Las siguientes líneas definen el footline.
\makeatletter
\setbeamertemplate{footline}
{
  \leavevmode%
  \hbox{%
    \begin{beamercolorbox}[wd=.26\paperwidth,ht=2.25ex,dp=1ex,left]{author in head/foot}%
      \hspace*{2em}\usebeamerfont{author in head/foot}\insertauthor
    \end{beamercolorbox}%
    \begin{beamercolorbox}[wd=.48\paperwidth,ht=2.25ex,dp=1ex,center]{title in head/foot}%
      \usebeamerfont{title in head/foot}\inserttitle
    \end{beamercolorbox}%
    \begin{beamercolorbox}[wd=.26\paperwidth,ht=2.25ex,dp=1ex,right]{date in head/foot}%
      \usebeamerfont{date in head/foot}
      \insertframenumber{} / \inserttotalframenumber\hspace*{2em} 
    \end{beamercolorbox}
  }%
  \vskip0pt%
}
\makeatother

% La siguiente línea elimina la barra de navegación
\setbeamertemplate{navigation symbols}{}

% La siguiente define que los bloques tengan bordes redondeados
\setbeamertemplate{blocks}[rounded]

% Las siguientes líneas definen el formato de los bloques por default y los de
% ejemplos
\setbeamercolor{block title}{use=structure,fg=white,bg=structure.fg!25!black}
\setbeamercolor{block body}{parent=normal text,use=block title,bg=block title.bg!10!bg}
\setbeamercolor{block title example}{fg=white,bg=green!25!black}
\setbeamercolor{block body example}{parent=normal text,bg=green!50!white}

% La siguiente línea hace que el texto del PDF resultante sea copiable
\usepackage[T1]{fontenc}

% La siguiente línea hace que sea posible escribir acentos y demás símbolos en UTF-8
\usepackage[utf8]{inputenc}

% La siguiente línea define la fuente
\usepackage{lmodern} \normalfont
\usepackage{tikz}
\usetikzlibrary{cd,positioning,decorations.text}

% La siguiente línea es para formatear el texto en columnas
\usepackage{multicol}

% Descomentar el siguiente comando para tener una carátula antes del inicio de cada sección.
%\AtBeginSection[]{
%  \begin{frame}[plain,noframenumbering]
%    \vfill \centering
%    \begin{beamercolorbox}[sep=8pt,center,shadow=true,rounded=true]{title}
%      \usebeamerfont{title}\insertsectionhead\par
%    \end{beamercolorbox} 
%    \vfill
%  \end{frame}
%}

% % % % % Título y autor (para la metadata del PDF)
\title{} 	% Si querés poner símbolos matemáticos, acá usá
			 	% algo así \texorpdfstring{$\alpha$}{alpha}
\author{}
% % % % % %

\begin{document}

% Las siguientes opciones al frame son para que no salga la barra de abajo ni
% la cuente en la numeración (es para la portada)
\frame[plain,noframenumbering]
{
  \centering
  \vfill

  {\LARGE\bf
    Inferencia bayesiana
  }

  \line(1,0){250}
  \vfill
  \vfill
  {\footnotesize
    \structure{\Large\bf\rmfamily Gustavo Landfried}\\[.5ex]
    {Becario doctoral}
  }
  \vfill

  \vfill
  % Para quienes tienen un director
%   {\footnotesize
%     \structure{\normalsize\bf\rmfamily Esteban Mocskos}\\[.5ex]
%     {Director}
%   }
  % Para quienes tienen dos directores (descomentar todo el bloque multicols)
  \begin{multicols*}{2}
    {\footnotesize
    \structure{\normalsize\bf\rmfamily Esteban Mocskos}\\[.5ex]
    {Director}
    }
    \\
    {\footnotesize
    \structure{\normalsize\bf\rmfamily Diego Slezak}\\[.5ex]
    {Codirector}
    }
  \end{multicols*}
  \vfill
  \vfill


  {\Large
    4to Día de la Investigación en Ciencias de la Computación\\
  }
  {\scriptsize
    18 de marzo de 2022
  }
  \vfill
  \vfill

  \includegraphics[scale=0.05, bb= 23in -3in 25in 0in]{img/exactas_uba.png}
  \includegraphics[scale=0.14]{img/icc-horizontal.jpg}
  \includegraphics[scale=0.32, bb= -0.7in -0.15in 0in 0in]{img/logo-dc.pdf}
}


\begin{frame}[plain]
 \begin{textblock}{160}(0,08) \centering
  \Large La ciencia tiene pretensión de verdad
\end{textblock}

 \begin{textblock}{80}(0,24) \centering
 \large Verdades formales 
\end{textblock}

\begin{textblock}{60}(10,36) \centering
 Validadas dentro de \\ sistemas axiomáticos cerrados 
 
 \\[0.2cm]
 
 \centering
 
 Sin incertidumbre
\end{textblock}

\begin{textblock}{80}(80,24) \centering
 \large Verdades empíricas
\end{textblock}

\begin{textblock}{60}(90,36) \centering
 Validadas dentro de \\ sistemas naturales abiertos 
 
 \\[0.2cm]
 
 \centering
 
 Con incertidumbre
\end{textblock}

\only<2>{
\begin{textblock}{160}(0,66) \centering
\large ¿Cuáles son entonces la verdades empíricas?
\end{textblock}
}
\end{frame}

\begin{frame}[plain]
  \begin{textblock}{80}(34,14)
 \Large \textcolor{black!50}{Sorpresa}
\end{textblock}

\begin{textblock}{47}(113,73.5)
\centering \Large \textcolor{black!5}{Supervivencia}
\end{textblock}

\begin{textblock}{80}(103,27)
\Large \textcolor{black!10}{Creencia}
\end{textblock}

\begin{textblock}{80}(44,61)
\Large  \textcolor{black!15}{Dato}
\end{textblock}


 %\vspace{2cm}brown
%\maketitle
\Wider[2cm]{
\includegraphics[width=1\textwidth]{../auxiliar/images/peligro_predador}
}
\end{frame}










\begin{frame}[plain]
 \begin{textblock}{160}(4,4)
 \centering
  \Large Acuerdos intersubjetivos
 \end{textblock}

 
 \only<1>{
  \begin{textblock}{80}(-10,20) \centering
 \scalebox{0.8}{
\tikz{ %
         \node[factor, minimum size=1cm] (p1) {} ;
         \node[factor, minimum size=1cm, xshift=1.5cm] (p2) {} ;
         \node[factor, minimum size=1cm, xshift=3cm] (p3) {} ;
         
         
         \node[const, above=of p1, yshift=0.1cm] (np1) {\Large $?$};
         \node[const, above=of p2, yshift=0.1cm] (np2) {\Large $?$};
         \node[const, above=of p3, yshift=0.1cm] (np3) {\Large $?$};
         } 
}
\end{textblock}
% 
}
 
\only<2>{
  \begin{textblock}{80}(-10,20) \centering
 \scalebox{0.8}{
\tikz{ %
         \node[factor, minimum size=1cm] (p1) {} ;
         \node[factor, minimum size=1cm, xshift=1.5cm] (p2) {} ;
         \node[factor, minimum size=1cm, xshift=3cm] (p3) {} ;
         
         
         \node[const, above=of p1, yshift=-0.05cm] (np1) {\Large $1/3$};
         \node[const, above=of p2, yshift=-0.05cm] (np2) {\Large $1/3$};
         \node[const, above=of p3, yshift=-0.05cm] (np3) {\Large $1/3$};
         } 
}
\end{textblock}
% 
}

\only<2->{
\begin{textblock}{100}(55,24) \centering
\large Principio de indiferencia \\
\normalsize
Primer acuerdo intersubjetivo en contextos de incertidumbre
\end{textblock}
}

 
\only<3>{
 \begin{textblock}{80}(-10,20) \centering
 \scalebox{0.8}{
\tikz{ %
         \node[factor, minimum size=1cm] (p1) {\includegraphics[width=0.05\textwidth]{../auxiliar/images/cerradura.png}} ;
         \node[det, minimum size=1cm, xshift=1.5cm] (p2) {\includegraphics[width=0.06\textwidth]{../auxiliar/images/dedo.png}} ;
         \node[factor, minimum size=1cm, xshift=3cm] (p3) {} ;
         
         
         \node[const, above=of p1, yshift=0.1cm] (np1) {\Large $?$};
         \node[const, above=of p2, yshift=0.1cm] (np2) {\Large $0$};
         \node[const, above=of p3, yshift=0.1cm] (np3) {\Large $?$};
         } 
}
\end{textblock}
}

\only<4->{
 \begin{textblock}{80}(-10,20) \centering
 \scalebox{0.8}{
\tikz{ %
         \node[factor, minimum size=1cm] (p1) {\includegraphics[width=0.05\textwidth]{../auxiliar/images/cerradura.png}} ;
         \node[det, minimum size=1cm, xshift=1.5cm] (p2) {\includegraphics[width=0.06\textwidth]{../auxiliar/images/dedo.png}} ;
         \node[factor, minimum size=1cm, xshift=3cm] (p3) {} ;
         
         
         \node[const, above=of p1, yshift=-0.05cm] (np1) {\Large $1/3$};
         \node[const, above=of p2, yshift=0.1cm] (np2) {\Large $0$};
         \node[const, above=of p3, yshift=-0.05cm] (np3) {\Large $2/3$};
         } 
}
\end{textblock}
}


\only<3->{
\begin{textblock}{80}(-10,36) \centering
Datos \\[0.5cm]

\only<4->{
 Modelo causal ``Monty Hall'' \\[0.3cm]
\scalebox{0.6}{
\tikz{        
    
    \node[latent] (d) {\includegraphics[width=0.10\textwidth]{../auxiliar/images/dedo.png}} ;
    \node[const,left=of d] (nd) {\Large $s$} ;
    
    \node[latent, above=of d, xshift=-1.5cm] (r) {\includegraphics[width=0.12\textwidth]{../auxiliar/images/regalo.png}} ;
    \node[const,left=of r] (nr) {\Large $r$} ;
    
    
    \node[latent, fill=black!30, above=of d, xshift=1.5cm] (c) {\includegraphics[width=0.12\textwidth]{../auxiliar/images/cerradura.png}} ;
    \node[const,left=of c] (nc) {\Large $c$} ;
         
    \edge {r,c} {d};
}
}
}
\end{textblock}
}


\only<5>{
\begin{textblock}{100}(55,54) \centering
\large Inferencia Bayesiana \\
\normalsize
Creencia previa que sigue siendo compatible con datos y modelo
\end{textblock}
}

\end{frame}




\begin{frame}[plain]
\begin{textblock}{160}(0,4)
\centering \Large Modelos causales alternativos
\end{textblock}



\begin{textblock}{160}(14,14) 
\begin{equation*}
 P(\text{Modelo}|\text{Datos}) = \frac{\only<1->{\overbrace{P(\text{Data}|\text{Modelo})}^{\text{\footnotesize Predicción a priori}}} \only<1->{P(\text{Modelo})} }{ P(\text{Data})} \phantom{\frac{\overbrace{P(\text{Datos}|\text{Modelo})}^{\text{Evidencia}}}{ P(\text{Datos})}}
\end{equation*}
\end{textblock}
% 
% \only<2>{
% \begin{textblock}{160}(0,47) 
% \begin{align*}
% P(\text{Data}|\text{Modelo}) & = \sum_{i} P(\text{Data}|\text{Hypothesis}_i,\text{Model}) P(\text{Hypothesis}_i|\text{Model}) 
% \end{align*}
% \end{textblock}
% }



\only<2>{

\begin{textblock}{140}(10,30) 
\centering
\includegraphics[width=0.66\textwidth]{figures/monty_hall_selection.pdf} \hspace{2cm}
\end{textblock}

\begin{textblock}{80}(86,30)
\centering
\scalebox{0.5}{
\tikz{        
    
    \node[latent] (d) {\includegraphics[width=0.10\textwidth]{../auxiliar/images/dedo.png}} ;
    \node[const,left=of d] (nd) {\Large $s$} ;
    
    \node[latent, above=of d, xshift=-1.5cm] (r) {\includegraphics[width=0.12\textwidth]{../auxiliar/images/regalo.png}} ;
    \node[const,left=of r] (nr) {\Large $r$} ;
    
    
    \node[latent, fill=black!30, above=of d, xshift=1.5cm] (c) {\includegraphics[width=0.12\textwidth]{../auxiliar/images/cerradura.png}} ;
    \node[const,left=of c] (nc) {\Large $c$} ;
         
    \edge {r,c} {d};
}
}
\end{textblock}


\begin{textblock}{80}(86,64)
\centering
\scalebox{0.5}{
 \tikz{            
    \node[latent,] (r) {\includegraphics[width=0.12\textwidth]{../auxiliar/images/regalo.png}} ;
    \node[const,left=of r] (nr) {\Large $r$} ;    
    
    
    \node[latent, below=of r] (d) {\includegraphics[width=0.10\textwidth]{../auxiliar/images/dedo.png}} ;
    \node[const, left=of d] (nd) {\Large $s$} ;

    \edge {r} {d};
    
}
}
\end{textblock}

}
\end{frame}

\begin{frame}[plain]
\begin{textblock}{160}(0,4)
\centering \Large Poster 1 \\  
\large 
Estimador de habilidad estado-del-arte
\end{textblock}

\begin{textblock}{140}(10,18) \centering
\includegraphics[width=\linewidth]{../static/atp.pdf}
\end{textblock}

\end{frame}


\begin{frame}[plain]
\begin{textblock}{160}(0,4)
\centering \Large Poster 2 \\  
\large 
Ventaja de la cooperación \textcolor{gray}{y la especialización}
\end{textblock}

\begin{textblock}{160}(0,14) \centering
\begin{equation*} \label{eq:modelo_exponencial}
f(\Aa) =
\begin{cases}
 1.5 & \Aa = \text{ Cara } \\
 0.6 & \Aa = \text{ Seca }
\end{cases}
\end{equation*}\\[-1.5cm]
\end{textblock}

\only<2->{
\begin{textblock}{80}(0,36) \centering
\includegraphics[width=0.90\linewidth]{../figures/pdf/ergodicity_individual_trayectories_y.pdf}
\end{textblock}
}

\only<3>{
\begin{textblock}{80}(80,36) \centering
\includegraphics[width=0.90\linewidth]{../figures/pdf/ergodicity_desertion.pdf}
\end{textblock}
}

\end{frame}



\end{document}
